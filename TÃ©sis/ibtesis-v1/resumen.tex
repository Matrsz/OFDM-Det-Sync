\begin{resumen}%

La recepción de señales inalámbricas en todo sistema de comunicaciones inicia con las etapas de detección y sincronismo, las cuales necesitan ejecutarse en línea durante la operación del sistema. En la transmisión empleando OFDM de acuerdo al estándar IEEE 802.11 se especifican las características que debe tener de una trama transmitida para permitir que el receptor implemente estas etapas. En el presente trabajo se describe el desarrollo e implementación de algoritmos de detección y sincronismo diseñados para operar en línea sobre un dispositivo SDR utilizando LabVIEW. 

En la primera etapa del trabajo se estudian las especificaciones para la transmisión con OFDM definidas por el estándar IEEE 802.11a, las cuales habilitan la detección y sincronismo por medio de la inclusión del preámbulo de capa física a las trama transmitidas. Además, se toma consideración de las especificaciones del estándar referentes a las escalas de tiempo y frecuencia usadas en la transmisión.

La segunda y la tercera etapa se centran en estudiar los algoritmos de sincronismo y detección respectivamente, los cuales que se desarrollan en base al conocimiento del preámbulo de capa física. Se verifica el correcto funcionamiento de los algoritmos y se procede a implementarlos en LabVIEW.

La etapa final consiste en la implementación de un sistema que integra los algoritmos implementados durante la etapa anterior, analizando posibles fuentes de error propias a la operación en línea y aplicando estrategias para evitar a las mismas. Se verifica una prueba de concepto del sistema diseñado, se prueba si éste es capaz de alcanzar la tasa de iteraciones mínima requerida para la operación en tiempo real, y se analizan los pasos futuros requeridos para lograr que éste diseño sea capaz de alcanzar la tasa de iteraciones necesaria.

\end{resumen}

\begin{abstract}%
This is the title in English:\\
The thesis must reflect the work of the student, including the chosen methodology, the results and the conclusions that those results allow us to draw.
\end{abstract}


%%% Local Variables: 
%%% mode: latex
%%% TeX-master: "template"
%%% End: 
